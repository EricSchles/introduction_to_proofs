\documentclass{book}
\title{Introduction to Higher Mathematics}
\author{Eric Schles}
\date{ Friday, December 19th, 20014}
\begin{document}
\maketitle
Welcome to Higher mathematics
\tableofcontents
\chapter{Why We Prove Things}
In this chapter I explain why proving things can be useful and how to make use of it in your daily life, career and in the larger context of the universe of knowledge
\chapter{Understanding numbers}
In this chapter I give an introduction to numbers and how they came about.  Then I give some basic examples and exercises of properties of numbers.  
\chapter{A First Technique}
In this chapter I give an analysis of a first technique for proving things as well as a number of examples of how to use the technique.
\chapter{Adding To Our List Of Techniques}
In this chapter I present more techniques with similar analysis.  Here we will review the first technique and show how it can be used in conjunction with the other techniques we will learn.
\chapter{The Axiomatic Approach}
Here we will discuss some major axioms in mathematics. As well as the use of conventions and primitives.  This will inform a larger understanding of how thought works. And provide a way to look beneath the covers of logic.
\chapter{Towards A First Theory}
Here we will use our axioms to construct a theory of mathematics and then we will relate this theory to some tanglible examples.
\chapter{Expanding Our Theory}
In this chapter we will look into building a second theory from a new set of axioms and then see how the two theories interweave.

\chapter{Exploring new roads}
In this chapter I will introduce you to category theory.  This will allow you to formalize the process of connecting two theories and even connecting many theories.  We will close out the chapter by looking at how to set up your own set of axioms and then try to connect this to existing axioms and conclusions.



\end{document}
